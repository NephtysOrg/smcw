\documentclass[10pt, compress]{beamer}

\usetheme{m}

\usepackage{booktabs}
\usepackage[scale=2]{ccicons}
\usepackage{minted}

\usepgfplotslibrary{dateplot}

\usemintedstyle{trac}

\title{Structures et modèles des contenus du web}
\subtitle{Conception et transformations de langages de balisage}
\date{\today}
\author{Charles Follet, Roland Bary}
\institute{Université de Pau et des Pays de l'Adour}

\begin{document}
\maketitle
\begin{frame}[fragile]
  \frametitle{Sommaire}
  \tableofcontents
\end{frame}

\section{Introduction}
\begin{frame}[fragile]
  \frametitle{Introduction}
  \begin{itemize}[<+->]
  \setbeamertemplate{itemize item}[square]
    \item{Evolution du web}
    \item{Necessité de structurer et modéliser les contenus du web de la manière la plus indépendante possible}
    \item{Manipulation de données hétérogènes voire irrégulières exprimés en XML}
    \item{Etude de cas: Conception de langages de balisages permettant la modélisation d'un type spécifique de données}
    \item{Documents relatifs à des circuits de randonnée}
  \end{itemize}
\end{frame}

\section{Les besoins à modéliser}
\begin{frame}[fragile]
  \frametitle{Expression des besoins à modéliser}
      \begin{itemize}
      \setbeamertemplate{itemize item}[square]
	\setbeamertemplate{itemize item}[square]
	\item{Informations de randonnée}
	\item{Hiérachie/Structure des documents}
	\item{Fiches d'information et fiches techniques}
      \end{itemize}    
\end{frame}
%%%%%%%%%%%%%%%%%%%%%%%%%%%%%%%%%%%%%%%%%%%%%%%%%%%%%%%%%%%%%%%%%%%%%%%%%%%%%%%%%%%%%%%%%%%%%%%%%%%%%%%%%%%%%%%%%%%%%%%%%
\section{Conception des langages de balisage}
\begin{frame}[fragile]
 \frametitle{Conception des langages de balisage}
  \begin{description}[<+->]
  \setbeamertemplate{itemize item}[square]
   \item [Balisage fil de l'eau/Balisage naturel] Aucune contrainte particulière 
   \item [Balisage type bd/Balisage structuré] Necessité de conserver la même organisation des balises
   \item [Problématique] Langage naturel vers Langage structuré $\Rightarrow$ Perte d'informations
   \item []\begin{minted}[fontsize=\small]{latex}
    tourner le dos au <lieu>gave de Pau</lieu> et prendre
    ...<lieu>gave de Pau</lieu>...\end{minted}
  \end{description}
\end{frame}

%%%%%%%%%%%%%%%%%%%%%%%%%%%%
\subsection{Hiérarchie}
\begin{frame}[fragile]
 \frametitle{Conception des langages de balisage - Hiérarchie}
  \begin{itemize}[<+->]
  \setbeamertemplate{itemize item}[square]
    \item{Modélisation des différentes parties qui constituent le document}
    \item{Conservation de l'intégralité du document}
    \item{Préambule, description, thematique, synthese, titres, sous-titres, listes,...}
    \item{Avantage: Structuration précise et spécifique (s'applique bien aux documents etudiés) }
    \item{Inconvénient: ne peut structurer que le jeu de documents (jurançon, 7villages, Crabere,..) , modélisation pas assez affinée}
  \end{itemize}
\end{frame}
%%%%%%%%%%%%%%%%%%%%%%%%%%%%
\subsection{Information de randonnées}
\begin{frame}[fragile]
 \frametitle{Conception des langages de balisage - Randonnée}
\end{frame}
%%%%%%%%%%%%%%%%%%%%%%%%%%%%
\subsection{Fiches techniques et informatives}
\begin{frame}[fragile]
 \frametitle{Conception des langages de balisage - Fiches}
 \begin{itemize}[<+->]
  \setbeamertemplate{itemize item}[square]
    \item{Modélisation de fiches recapitulatives}
    \item{Informations retenues: Fiche technique et fiche d'information (listes d'item)}
    \item{}
    \item{Avantage: }
    \item{Inconvénient: On ne peut modéliser que deux types de fiches}
  \end{itemize}
\end{frame}
%%%%%%%%%%%%%%%%%%%%%%%%%%%%%%%%%%%%%%%%%%%%%%%%%%%%%%%%%%%%%%%%%%%%%%%%%%%%%%%%%%%%%%%%%%%%%%%%%%%%%%%%%%%%%%%%%%%%
\section{Transformations des langages}
\begin{frame}[fragile]
 \frametitle{Transformations des langages}
\end{frame}

\subsection{Transformation sans contrôle}
\begin{frame}[fragile]
 \frametitle{Transformations des langages - sans contrôle}
\end{frame}

\subsection{Transformation avec contrôle}
\begin{frame}[fragile]
 \frametitle{Transformations des langages - avec contrôle}
\end{frame}
%%%%%%%%%%%%%%%%%%%%%%%%%%%%%%%%%%%%%%%%%%%%%%%%%%%%%%%%%%%%%%%%%%%%%%%%%%%%%%%%%%%%%%%%%%%%%%%%%%%%%%%%%%%%%%%%%%%%%
\section{Conclusion}
\begin{frame}[fragile]
\end{frame}

\end{document}
