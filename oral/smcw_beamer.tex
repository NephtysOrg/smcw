\documentclass[10pt, compress]{beamer}

\usetheme{m}

\usepackage{booktabs}
\usepackage[scale=2]{ccicons}
\usepackage{minted}

\usepgfplotslibrary{dateplot}

\usemintedstyle{trac}

\title{Structures et modèles des contenus du web}
\subtitle{Conception et transformations de langages de balisage}
\date{\today}
\author{Charles Follet, Roland Bary}
\institute{Université de Pau et des Pays de l'Adour}

\begin{document}
\maketitle
\begin{frame}[fragile]
  \frametitle{Sommaire}
  \tableofcontents
\end{frame}

\section{Introduction}
\begin{frame}[fragile]
  \frametitle{Introduction}
  \begin{itemize}[<+->]
  \setbeamertemplate{itemize item}[square]
    \item{Evolution du web}
    \item{Necessité de structurer et modéliser les contenus du web de la manière la plus indépendante possible}
    \item{Manipulation de données hétérogènes voire irrégulières exprimés en XML}
    \item{Etude de cas: Conception de langages de balisages permettant la modélisation d'un type spécifique de données}
    \item{Documents relatifs à des circuits de randonnée}
  \end{itemize}
\end{frame}

\section{Les besoins à modéliser}
\begin{frame}[fragile]
  \frametitle{Expression des besoins à modéliser}
      \begin{itemize}
      \setbeamertemplate{itemize item}[square]
	\setbeamertemplate{itemize item}[square]
	\item{Informations de randonnée}
	\item{Hiérachie/Structure des documents}
	\item{Fiches d'information et fiches techniques}
      \end{itemize}    
\end{frame}
%%%%%%%%%%%%%%%%%%%%%%%%%%%%%%%%%%%%%%%%%%%%%%%%%%%%%%%%%%%%%%%%%%%%%%%%%%%%%%%%%%%%%%%%%%%%%%%%%%%%%%%%%%%%%%%%%%%%%%%%%
\section{Conception des langages de balisage}
\begin{frame}[fragile]
 \frametitle{Conception des langages de balisage}
  \begin{itemize}[<+->]
  \setbeamertemplate{itemize item}[square]
   \item {Balisage fil de l'eau/Balisage naturel}
    \begin{itemize}
     \item Avantage : Aucune contrainte particulière
     \item Inconvénients: Difficultés rencontrées sur le traitement de certaines données (requêtes de type bd)
    \end{itemize}
   \item {Balisage type bd/Balisage structuré Necessité de conserver la même organisation des balises}
    \begin{itemize}
     \item Avantage: Possibilité de faire aisément des requêtes de type BD
     \item Inconvénients: Perte d'information, contrainte de structure
    \end{itemize}
   \item {Problème :Langage naturel vers Langage structuré $\Rightarrow$ Perte d'informations}
   \item Exemple: tourner le dos au <lieu>gave de Pau</lieu> et prendre
   $\Rightarrow$ ...<lieu>gave de Pau</lieu>...
  \end{itemize}
\end{frame}

%%%%%%%%%%%%%%%%%%%%%%%%%%%%%%%%%%%%%%%%%%%%%%%%%%%%%%%%%%%%%%%%%%%%%%%%%%%%%%%%%%%%%%%%%%%%%%%%%%%%%%%%%%%%%%%%%%%%%%%%%%%
\subsection{Hiérarchie}
\begin{frame}[fragile]
 \frametitle{Conception des langages de balisage - Hiérarchie}
  \begin{itemize}[<+->]
  \setbeamertemplate{itemize item}[square]
    \item{Modélisation des différentes parties qui constituent le document}
    \item{Conservation de l'intégralité du document}
    \item{Préambule, description, thematique, synthese, titres, sous-titres, listes,...}
    \item{Avantage: Structuration précise et spécifique (s'applique bien aux documents etudiés) }
    \item{Inconvénient: ne peut structurer que le jeu de documents (jurançon, 7villages, Crabere,..) , modélisation pas assez raffinée}
  \end{itemize}
\end{frame}
%%%%%%%%%%%%%%%%%%%%%%%%%%%%%%%%%%%%%%%%%%%%%%%%%%%%%%%%%%%%%%%%%%%%%%%%%%%%%%%%%%%%%%%%%%%%%%%%%%%%%%%%%%%%%%%%%%%%%%%%%%%%
\subsection{Information de randonnées}
\begin{frame}[fragile]
 \frametitle{Conception des langages de balisage - Randonnée}
 \begin{itemize}[<+->]
  \setbeamertemplate{itemize item}[square]
    \item{Modélisation des différentes étapes qui constitues la randonnée}
    \item{L'intégralité du document n'est pas conservée}
    \item{Lieux,description,distance,durée,...}
    \item{Avantage: Permet des recherches temporelles et géographiques (example : Randonnée de moins d'une heure)}
  \end{itemize}
\end{frame}
%%%%%%%%%%%%%%%%%%%%%%%%%%%%
\subsection{Fiches techniques et informatives}
\begin{frame}[fragile]
 \frametitle{Conception des langages de balisage - Fiches}
 \begin{itemize}[<+->]
  \setbeamertemplate{itemize item}[square]
    \item{Modélisation de fiches recapitulatives}
    \item{Informations retenues: Fiche technique et fiche d'information (listes d'item)}
    \item{fiches technique, fiche d'information}
    \item{Avantage: Permet un résumé rapide, ajoute des information supplémentaires aux autres langages}
    \item{Inconvénient: On ne peut modéliser que deux types de fiches}
  \end{itemize}
\end{frame}
%%%%%%%%%%%%%%%%%%%%%%%%%%%%%%%%%%%%%%%%%%%%%%%%%%%%%%%%%%%%%%%%%%%%%%%%%%%%%%%%%%%%%%%%%%%%%%%%%%%%%%%%%%%%%%%%%%%%
\section{Transformations des langages}
\subsection{Transformation sans contrôle}
\begin{frame}[fragile]
 \frametitle{Transformations des langages - sans contrôle}
 \begin{itemize}[<+->]
  \setbeamertemplate{itemize item}[square]
    \item Langage Fil de l'eau $\blacktriangleright$ langage BD : XSLT version 1
    \item Avantage: Pas de contraintes à respecter, large choix de transformations possibles 
    \item Inconvénient: Inexactitude quand a la conservation des données depuis l'entrée vers la sortie
  \end{itemize}
\end{frame}

\subsection{Transformation avec contrôle}
\begin{frame}[fragile]
 \frametitle{Transformations des langages - avec contrôle}
 \begin{itemize}[<+->]
  \setbeamertemplate{itemize item}[square]
    \item{lors du passage du langage de type BD vers l'application de sortie \LaTeX}
    \item{Contrôle effectué en entrée et pas en sortie (En sortie contrôle par la compilation \LaTeX}
    \item{Contrôle sur le contenu et les attributs des balises par les schemas XML}
    \item{Contrôle sur les attributs des balises également grâce aux DTDs}
    \item Limites: 
      \begin{itemize}
       \item Etablissement d'un ordre sur une séquence de valeurs numériques en occurence sur les identifiants des différentes étapes d'un circuit de randonnée  
       \item XML Schema généré à partir d'une DTD avec une erreur d'ambiguïté quand la DTD contient des élément admettant plusieurs définitions possibles; Solution: Ecriture plus complexe de la DTD
      \end{itemize}
  \end{itemize}
\end{frame}
%%%%%%%%%%%%%%%%%%%%%%%%%%%%%%%%%%%%%%%%%%%%%%%%%%%%%%%%%%%%%%%%%%%%%%%%%%%%%%%%%%%%%%%%%%%%%%%%%%%%%%%%%%%%%%%%%%%%%
\section{Conclusion}
\begin{frame}[fragile]
   \begin{itemize}[<+->]
  \setbeamertemplate{itemize item}[square]
    \item Découverte de la véritable utilisation du meta-langage XML et des multiples possibilités de conception de langages.
    \item Limites rencontrées quand à l'utilisation des DTD pour structurer l'information.
    \item Utilisation de schémas XML pour palier à cette limite.
  \end{itemize}
\end{frame}
\end{document}
