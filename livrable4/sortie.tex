\documentclass{article}
\usepackage[utf8]{inputenc}
\usepackage[francais]{babel}
\usepackage{hyperref}\usepackage{xcolor}
\newcommand\ytl[2]{
\parbox[b]{8em}{\hfill{\color{cyan}\bfseries\sffamily #1}~$\cdots\cdots$~}\makebox[0pt][c]{$\bullet$}\vrule\quad \parbox[c]{4.5cm}{\vspace{7pt}\color{red!40!black!80}\raggedright\sffamily #2.\\[7pt]}\\[-3pt]}
\begin{document}\section{}

        \begin{table}[h]

        \centering

        \begin{minipage}[t]{.7\linewidth}

        \color{gray}

        \rule{\linewidth}{1pt}

        (1)
\ytl{1220 m}{Avenue Joliot-Curie \\(00h m )}
(2)
\ytl{2300 m}{Le Clos Touzet \\(00h m )}
(3)
\ytl{1200 m}{Le chemin Mirabel \\(00h m )}
(4)
\ytl{2430 m}{Le Clos Lapeyre \\(00h m )}
(5)
\ytl{3060 m}{Chemin de la Rec \\(00h m )}
(6)
\ytl{3915 m}{Les coteaux de St-Faust face. \\(01h m )}
(7)
\ytl{3100 m}{Le ruisseau des Hies descente humide. \\(00h m )}
(8)
\ytl{4770 m}{Le gave de Pau \\(01h m )}

            \bigskip

            \rule{\linewidth}{1pt}

            \end{minipage}

            \end{table}

        (1)\indent Il faut tourner le dos au gave de Pau et prendre la rue
                                de l'Artisanat vers le sud. Traversez la double-voie et poursuivez
                                en face par l'avenue Joliot-Curie. Laissez à droite le lycée
                                professionnel de Jurançon tout en métal brillant et en tubes
                                métalliques puis franchissez une passerelle pour pénétrer dans une
                                zone de pavillons modernes. Poursuivez par la rue Auguste Renoir
                                puis, au bout, virer à droite en direction de la rocade.
                            \\(2)\indent Traverser la rocade puis continuez en vous élevant par
                                le chemin de Monplaisir, à la limite d'un bois. Laisser les
                                bâtiments du « Nid béarnais » à gauche et continuer en sous-bois.
                                Débouchez au sommet du coteau face « Au Clos Touzet » où la vue sur
                                la plaine de Pau mérite un long coup d'oeil. Poursuivez en face. Ne
                                pas descendre trop longtemps vers Jurançon, mais suivre l'avenue des
                                Frères Barthélemy en face. \\(3)\indent Face à l'entrée du Clos Henri IV, au N 223, descendre à
                                droite le chemin Mirabel (panneau) qui se faufile en sous-bois.
                                Juste avant l'entrée de la ferme Larrau (remarquer le séquoia isolé)
                                prenez un chemin creux à gauche qui plonge dans un sous-bois
                                magnifique. Il décrit une épingle et s'enfonce dans une forêt où la
                                lumière pénètre difficilement. Rejoignez le lit d'un petit ruisseau
                                que vous suivez vers l'aval. Mousses, lichens, fougères constituent
                                le décor luxuriant de cette descente humide. \\(4)\indent Emprunter la route à gauche (chemin Vignau) où vous
                                rencontrez un chemin équestre (itinéraire équestre Pyrénéen) que
                                vous suivrez maintenant facilement vers le sud. Franchissez un pont
                                et grimpez une piste cimentée « camin de Sobolle ». Dépasser la
                                belle ferme Sobolle et continuer dans le bois. Rejoignez un chemin
                                goudronné et le suivre à gauche vers le Clos Lapeyre. Après la
                                première vigne, descendre à droite. Au premier palier, continuer à
                                droite entre les vignes. En dessous, descendez en lisière d'un petit
                                bois et franchissez un portillon métallique. Emprunter la D230 à
                                droite sur 1 km. \\(5)\indent Au croisement de Roumiga, tourner à gauche, franchissez
                                un pont et traversez la vallée vers le nord dans une cuvette
                                profonde, agréable et paisible. Prenez à droite la D 217, puis aller
                                à gauche vers St-Faust (2 km). Grimper le chemin de la Rec qui
                                serpente en sous-bois et rejoignez la commune de St-Faust face à
                                l'église (belle vue sur l'Ossau). \\(6)\indent Emprunter la D 502 vers la droite pendant 1500 m et
                                passer devant la Cité des Abeilles. Prenez le chemin Mantoulan à
                                gauche et longer de belles vignes plantées sur un flanc sud. Suivez
                                le balisage d'un chemin baptisé « les coteaux de St-Faust». Au
                                sommet de la colline, prenez le beau chemin Constantine à droite qui
                                file vers le nord. Il contourne deux exploitations (Larréheuga et
                                Coutchous) et plonge entre les prairies. Rejoignez un chemin
                                goudronné. En bas de la côte, virer à droite le long du ruisseau
                                Lahourcade. \\(7)\indent Franchir le ruisseau à gué et tourner à droite (Panneau
                                interdiction aux quads). S'élever rudement par un sentier dans les
                                bois. Rejoignez l'église de St-Faust-de-bas, puis descendez à gauche
                                par la D 502. Dans un virage prononcé à gauche, partir à droite à
                                travers bois par le chemin Coureyou, puis rejoignez une nouvelle
                                fois la D 502 qui conduit à l'entrée de Laroin. La quitter pour
                                suivre le ruisseau des Hies en face. Franchir une passerelle et
                                parvenez au centre du village. Traverser la D 2 et utiliser le
                                chemin du Stade qui rejoint le ruisseau des Hies face à une
                                passerelle. Longer le ruisseau puis passez sous la rocade D 2 par un
                                tunnel à l'ambiance curieuse. \\(8)\indent Suivre la rive gauche du gave de Pau longuement où les
                                amateurs d'oiseaux se régaleront en surprenant les hérons ou les
                                cormorans. Laissez un centre équestre à gauche et ignorez juste
                                après la remarquable passerelle piétons de Laroin aux haubans en
                                fibre de verre. Rejoindre le pont de Billères où vous passerez
                                dessous et l'emprunter pour franchir le gave. Prendre à droite à son
                                extrémité est un sentier facile et continuer maintenant rive droite
                                en longeant le golf de Billères par le sud. Contourner le restaurant
                                "Au bord de l'eau" et franchissez la passerelle pour piétons afin de
                                retrouver le parking de départ. \\\end{document}